\documentclass[a4paper,11]{article}
\usepackage[utf8]{inputenc}
\usepackage{graphicx}
\graphicspath{ {./images/} }

\title{Computational Physics Laboratory report}
\author{Stefano Ge}
\date{Winter semester 2025}

\begin{document}

\maketitle

\tableofcontents
{}

%--------------------------------------------------------------------------
%--------------------------------------------------------------------------
\section{Error analysis}
%--------------------------------------------------------------------------
%--------------------------------------------------------------------------


\subsubsection{Approximating exponentials}

\paragraph{Results}

\begin{figure}[htbp]
    \centering
    \includegraphics[width=\textwidth]{approx_exp1_0.pdf}
    \caption{Truncation errors when approximating $e^{x}$ through a 
				truncated power series. $N$ is the degree of the polinomyal used
				for the approximation. The plot of $x^{N+1}/(N+1)$ has been
				added for reference.}
    \label{fig:approx_exp}
\end{figure}

\paragraph{Remarks}

\begin{itemize}
	\item{Horner's method was used in polynomial evaluation, because it is
			faster and reduces roundoff errors. A naive implementation of a 
			polynomial computation requires more operations (\(O(N^{2})\)) and, 
			for small values of $x$, adds together numbers of order 1 and $x^{N}$,
			while Horner's method is \(O(N)\) and keeps the additions only
			between terms of order 1 and x.}
	\item{By Taylor's theorem, if \(\hat{f}(x)\) is the power series
			of $e^{x}$ truncated after the N-th term, the error is of order 
			$O(x^{N+1})$ as x approaches 0. Computing an additional term of the
			series shows that:
			\[|e^{x}-\hat{f}(x)|=\frac{x^{N+1}}{(N+1)!}+o(x^{N+1})\]
			It is then not surprising that the 
			approximation error in Figure \ref{fig:approx_exp} is 
			indistinguishable from $x^{N+1}/(N+1)$ for small values of x. For
			larger values, the contribution of higher order terms in the series 
			becomes ever more important, and hence the divergence from
			$x^{N+1}/(N+1)$.}
			
\end{itemize}


%-------------------------------------
\setcounter{subsection}{1}
\subsection{Floating-point arithmetic and roundoff errors}
%-------------------------------------

% it might prove useful: https://en.wikipedia.org/wiki/Machine_epsilon

\subsubsection{Computing the Basel problem}

\paragraph{Results}

\begin{figure}[htbp]
    \centering
    \includegraphics[width=\textwidth]{basel1_2_1.pdf}
    \caption{Truncation error when computing \(pi^{2}/6\) with the Basel problem
		infinite series, with 32bit floating precision.}
    \label{fig:basel}
\end{figure}

\paragraph{Remarks}

%-------------------------------------
\subsection{Error propagation and condition number}
%-------------------------------------

\subsubsection{Computing statistical momenta}

\subsubsection{Condition number: study of a simple algorithm}

%--------------------------------------------------------------------------
%--------------------------------------------------------------------------
\section{Linear systems}
%--------------------------------------------------------------------------
%--------------------------------------------------------------------------

test text

\subsection{Forward- and back-substitution}

\subsection{LUP Decomposition }

This is test text

\subsection{}

%--------------------------------------------------------------------------
%--------------------------------------------------------------------------
\section{Interpolation}
%--------------------------------------------------------------------------
%--------------------------------------------------------------------------


%--------------------------------------------------------------------------
%--------------------------------------------------------------------------
\section{Roots of nonlinear equations}
%--------------------------------------------------------------------------
%--------------------------------------------------------------------------



%--------------------------------------------------------------------------
%--------------------------------------------------------------------------
\section{Numerical integration}
%--------------------------------------------------------------------------
%--------------------------------------------------------------------------

%-------------------------------------
\subsection{Newton-Cotes formula}
%-------------------------------------


\subsubsection{Trapezoidal rule}


test


\subsubsection{Simpson's rule}


test

%-------------------------------------
\subsection{Free-nodes integration}
%-------------------------------------


\subsubsection{Nodes and weights of Gauss-Legendre rule}

something

\paragraph{Remarks}

\subsubsection{Integrals with Gauss-Legendre rule}

%-------------------------------------
\subsection{Advanced topics in integration}
%-------------------------------------

%--------------------------------------------------------------------------
%--------------------------------------------------------------------------
\section{Ordinary differential equations}
%--------------------------------------------------------------------------
%--------------------------------------------------------------------------


\end{document}
